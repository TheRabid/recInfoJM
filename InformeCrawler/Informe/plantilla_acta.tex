\documentclass[a4paper]{article}

\usepackage[utf8]{inputenc}
\usepackage[spanish]{babel}
\usepackage{graphics}
\usepackage{caption}
\usepackage{subcaption}
\usepackage[demo]{graphicx}
\usepackage{enumitem}
\usepackage{longtable}
\usepackage{listings}
\usepackage{listingsutf8}
\usepackage{framed}
\usepackage{float}
\usepackage{hyperref}

\begin{document}

\title{Comparativa de resultados entre el sistema de recuperación de información tradicional y Solr con la configuración por defecto}
\author{
	Jaime Ruiz-Borau Vizárraga\\
	\texttt{546751}
	\and
	Alberto Sabater Bailón\\
	\texttt{546297}
	}
\date{}
\maketitle

\section{Necesidades de información}
\subsection{Necesidad 1: Me interesan los trabajos en relación con el mundo de la musica y el sonido cuyo autor o director se llame Javier}
\paragraph{}Para esta necesidad de información, nuestro sistema de recuperación no ha devuelto \textbf{ningún resultado}.
\paragraph{}En cambio, el sistema de recuperación Apache Solr, ha devuelto 2015 documentos, cuyos 10 primeros resultados son los siguientes: 
\subsection{Necesidad 2: Estoy interesado en el periodo histórico de la Guerra de Independencia	(1808-1814). Me gustaría saber qué trabajos o artículos hay relacionados con este hecho englobados en la historia de España}
\paragraph{}Para esta necesidad de información, nuestro sistema de recuperación ha devuelto \textbf{9 documentos}: 7398, 7780, 7929, 7472, 7825, 7371, 7790, 7271 y 7301, en ese orden.
\paragraph{}En cuanto al análisis de la calidad de estos documentos, el primer documento es \textbf{bastante relevante} para la consulta, pero el segundo ya no resulta tan relevante y a partir del tercero dejan de tener algo que ver con la consulta de forma global (coinciden en palabras clave pero no tienen que ver con ella).

\paragraph{}Adicionalmente existe una tercera clase, \textit{CustomSpanishAnalyzer}, que extiende a la clase de Lucene \textbf{Analyzer}. Reimplementa el método \textit{createComponents}, que permite modificar las StopWords que suprimirá el analizador a la hora de lematizar una consulta.
\\
\newline Como detalle adicional, ambas clases Buscador e Indexador poseen un booleano \textbf{DEBUG} que a no ser que sea establecido manualmente a \textbf{true} impedirá que ambos programas muestren salida alguna por pantalla.

\section{Técnicas empleadas}
\subsection{Técnicas empleadas en Indexador}
\paragraph{}Dada la naturaleza de la colección de documentos del Zaguán, se ha optado por realizar un análisis de las etiquetas de los documentos extrayendo la información de todas ellas siguiendo el estándar de la iniciativa Open Archives \textbf{(Open Archives Initiate)}. Dicho estándar establece una serie de etiquetas en documentos XML: \textbf{Title} \textit{(título)}, \textbf{Creator} \textit{(autor)}, \textbf{Subject} \textit{(materia)}, \textbf{Description} \textit{(descripción)}, \textbf{Publisher} \textit{(editor)}, \textbf{Contributor} \textit{(asistente)}, \textbf{Date} \textit{(fecha)},
\textbf{Type} \textit{(tipo)}, \textbf{Format} \textit{(formato)}, \textbf{Identifier} \textit{(identificador)}, \textbf{Source} \textit{(fuente)}, \textbf{Language} \textit{(idioma)}, \textbf{Relation} \textit{(relacion)}, \textbf{Coverage} \textit{(cobertura)} y \textbf{Rights} \textit{(derechos)}.
\paragraph{}Aunque varios de ellos no aparecen en la colección de documentos del Zaguán, se implementaron igualmente su análisis y extracción en el caso de que en el futuro se llegasen a incluir dichas etiquetas en los documentos del Zaguán (mayor modularidad futura).
\paragraph{}Para cada etiqueta descrita anteriormente, simplemente se crearon los campos asociados con el mismo nombre y lo siguientes tipos:
\begin{itemize}
	\item \textbf{TextField} - Title, Creator, Subject, Description, Publisher y Contributor
	\item \textbf{IntField} - Date
	\item \textbf{StringField} - Type, Format, Identifier, Source, Language, Relation, Coverage y Rights
\end{itemize}
\paragraph{}Además de este análisis y extracción de las etiquetas, Indexador no emplea el típico Analyzer de Lucene, emplea el \textbf{CustomSpanishAnalyzer} que se mencionó en el diagrama de clases y que se explica en el apartado que describe las técnicas empleadas en la clase Buscador.
\\
\newpage
\subsection{Técnicas empleadas en Buscador}
\paragraph{}Para la clase Buscador, se han empleado las siguientes técnicas y metodologías a la hora de realizar una consulta en base a una necesidad de información:
\begin{itemize}
	\item \textbf{Correcto espaciado de la necesidad de información}:
	\\ Tal y como se proporcionó el fichero XML de necesidades de información, existían saltos de línea y tabuladores innecesarios que se optó por suprimir antes de procesar la necesidad para convertirla en consulta. 
	\item \textbf{Lematización de la consulta}:
	\\ Utilizando el CustomSpanishAnalyzer mencionado anteriormente, toda la necesidad de información queda lematizada a las raíces más importantes de cada palabra, suprimiendo palabras innecesarias (StopWords). 
	\paragraph{}Concretamente, el CustomSpanishAnalyzer diseñado, hereda las StopWords del \textbf{SpanishAnalyzer} por defecto de Lucene, y añade las siguientes: \textit{'interes', 'trabaj', 'llam', 'relacion', 'cuy', 'llam', 'period', 'gust', 'sab', 'articul', 'relacionad', 'hech', 'englob', 'tesis', 'trat', 'quier', 'document', 'encontr', 'situ', 'element', 'pertenezc'} y \textit{'mund'}.
	\paragraph{}Estas StopWords personalizadas fueron resultado de analizar individualmente cada consulta y eliminar las lematizaciones que se consideraron innecesarias para la consulta.
	\item \textbf{Pre-procesamiento de la necesidad de información}:
	\\Tras la lematización, se implementaron las siguientes técnicas antes de que el buscador procesase la consulta de forma general:
	\\
	\newline \textbf{1.-Búsqueda de autores en la necesidad:} Si el usuario en su necesidad de información especificó la palabra 'autor' y ésta, en algún momento de la misma necesidad, venía seguida de alguna otra palabra que comenzase por mayúscula, esta técnica asumiría que dicha palabra al comenzar por letra mayúscula es el \textbf{nombre del autor} del que se buscan documentos.
	\paragraph{}Por tanto, tras obtener el nombre del autor de los documentos que se buscan, se añadiría a la consulta final (de ahora en adelante \textbf{query}) una consecucion de TermQuerys (un tipo concreto de Querys en Lucene para términos) que \textbf{buscarán que haya coincidencias en el campo 'creator' para dicho nombre de autor}.
	Una vez hecho esto, se quita de la consulta lematizada final los términos autor y los nombres propios considerados.
	\\
	\newpage\textbf{2.-Búsqueda de períodos de años:} En ocasiones, el usuario especifica en su necesidad de información un período de años que buscar, bien sea el período de años entre los que pretende encontrar un trabajo o un período de años histórico sobre los que pretende encontrar un trabajo que hable de ellos.
	\paragraph{}Por ende, se ha implementado en la clase Buscador que busque en la consulta lematizada números de 4 cifras consecutivos y los considere un período de tiempo, de tal forma que a la query final se le añadirán unas \textbf{RangeQuerys} (querys para la comprobación de un rango de números sobre campos) sobre el campo \textbf{date} si dicho período de años es mayor o igual a 2008, fecha a partir de la cual existen documentos en el Zaguán. De esta forma la consulta se reducirá a documentos que se hallen dentro de dicho período especificado.
	\paragraph{}Para el resto de períodos, no se añaden querys de ningún tipo en particular, puesto que al final lo que queda de la consulta lematizada se especificará que se busque en los campos \textbf{description} y \textbf{title}. Se retirarán de la consulta lematizada los períodos de tiempo que sean mayores de 2008.
	\\
	\newline \textbf{3.-Búsqueda de la expresión 'últimos n años':} Como caso particular de las necesidades de información especificadas, se ha tomado en consideración el caso de que el usuario especifique en su necesidad de información que desea encontrar documentos cuya fecha de creación se halle entre los últimos n años.
	\\ 
	\newline Es por ello que se ha decidido buscar en la necesidad lematizada los términos 'últimos', un entero (n) y 'años', tras lo cual sabiendo el entero n y el año actual se puede generar una \textbf{RangeQuery} sobre el campo \textbf{date} que restringa los documentos cuya creación no esté comprendida en el rango descrito. Si se encuentran dichos términos, se retirarán de la consulta lematizada final.
	\\
	\newline \textbf{4.-Procesamiento de la consulta lematizada final:} Finalmente, tras todo el procesamiento expuesto anteriormente, la consulta lematizada restante es procesada de manera estándar. Se generarán TermQuerys para cada término que quede en la lista y se buscarán tanto en los campos \textbf{title} como \textbf{description}.
\end{itemize}
\newpage
\subsection{Algoritmo empleado para el ranking}
\paragraph{}El algoritmo utilizado para el cálculo del ranking, es el que utiliza Lucene por defecto. Dicho algoritmo usa una combinación del modelo de \textbf{espacio vectorial (VSM) y el modelo booleano} para determinar lo relevante que es un fichero.
\paragraph{}La idea general es que cuanto más veces aparece cada término de una query en un documento respecto del número de veces que ese término aparece en el resto de documentos de la colección, más relevante es ese documento para esa query.
Usa el modelo booleano para reducir el número de documentos devueltos haciendo uso de la lógica booleana en la specificación de la query.
\section{Resultados obtenidos}
\textbf{1.-'Me interesan los trabajos en relación con el mundo de la musica y el sonido cuyo autor o director se llame Javier.'}
\paragraph{}El resultado de esta consulta únicamente devuelve 2 ficheros, que son altamente relevantes. Ambos tratan de música y en ambos, 'Javier' es su autor. El resultado es tan breve porque únicamente se muestran los ficheros cuyo autor es Javier.
\paragraph{} \textbf{2.-'Estoy interesado en el periodo histórico de la Guerra de Independencia (1808-1814). Me gustaría saber qué trabajos o artículos hay relacionados con este hecho englobados en la historia de España'}
\paragraph{}La ejecución de esta consulta devuelve 192 documentos. Los primeros si que hacen referencia al periodo histórico de la Guerra de la Independencia de España, luego aparecen textos relativos a otras guerras o periodos históricos de España.

\paragraph{}\textbf{3.-'Tesis que traten de energías renovables, en el período de 2010 a 2015.'}
\paragraph{}La ejecución de esta consulta, devuelve 67 documentos. Todos ellos hablan en mayor o menor medida de las energías, siendo los primeros los que hacen más incapié en las energías renovables. Todos los documentos sevueltos, han sido publicados entre el año 2010 y 2015.

\paragraph{}\textbf{4.-'Quiero documentos sobre desarrollo de videojuegos o diseño de personajes en los últimos 5 años.'}
\paragraph{}La ejecución de esta consulta devuleve 490 documentos, de los cuales, os primeros si que hacen referencia al mundo de los videojuegos, el resto hacen referencia a temas de diseño o cualquier otro desarrollo. Todos los documentos han sido publicados en los últimos 5 años.

\paragraph{}\textbf{5.-'Me gustaría encontrar construcciones arquitectónicas situadas en España con elementos decorativos que pertenezcan tanto a la Edad Media como a la época gótica y cuyo estado de conservación sea óptimo.'}
\paragraph{}La ejecución de esta consulta devuelve 329 documentos, de temas muy dispares, siendo los primeros los centrados en temas de arquitectura o Edad Media.

\section{Referencias y documentación adicional}
\begin{itemize}
	\item Estándar Open Archives Initiative
	\newline \url{https://www.openarchives.org/}
	\item Algoritmo de ranking empleado por Lucene
	\newline \url{https://lucene.apache.org/core/3_5_0/scoring.html}
\end{itemize}
\end{document}