\documentclass[a4paper]{article}

\usepackage[utf8]{inputenc}
\usepackage[spanish]{babel}
\usepackage{graphics}
\usepackage{caption}
\usepackage{subcaption}
\usepackage[demo]{graphicx}
\usepackage{enumitem}
\usepackage{longtable}
\usepackage{listings}
\usepackage{listingsutf8}
\usepackage{framed}
\usepackage{float}
\usepackage{hyperref}

\begin{document}

\title{Evaluación del Sistema de Recuperación de Información Tradicional}
\author{
	Jaime Ruiz-Borau Vizárraga\\
	\texttt{546751}
	\and
	Alberto Sabater Bailón\\
	\texttt{546297}
	}
\date{}
\maketitle
\section{Informe}
\paragraph{1.- ¿Se corresponden las medidas de evaluación globales de tu sistema con alguna de las columnas que se muestran en el cuadro 2, o las curvas globales de precisión-exhaustividad de la figura 1? ¿Con qué columna/curva?}
\paragraph{}Si, se corresponden con las medidas de la curva n.
\paragraph{2.- ¿Qué decisiones de diseño de tu sistema (campos indexados, analizadores elegidos, modelo de ranking, procesamiento especial de consultas, ...) justifican las medidas obtenidas? ¿Hay otros factores externos que pueden haber afectado a estos resultados?}
\paragraph{}
\paragraph{3.- Según tu opinión, ¿Qué medida, o medidas, reflejan mejor la calidad de un sistema de recuperación?}
\paragraph{}La precisión@10, porque únicamente tiene en cuenta los primeros resultados. En un buscador, el usuario, por norma general, únicamente mira si los primeros resultados son buenos.
También el F1 score, porque es una medida que tiene en cuenta precisión y recall.
\paragraph{4.- En base a las medidas del cuadro 2, las curvas de la figura 1 y tu respuesta a la cuestión anterior. ¿Qué sistema que no sea el tuyo recomendarías utilizar?}
\paragraph{}Recomendaríamos \textbf{el sistema c} en base a los criterios expuestos en la pregunta 3.
\end{document}